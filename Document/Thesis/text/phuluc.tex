\chapter{Code chương trình}

Nếu báo cáo không có phụ lục, sinh viên tìm và xóa dòng \verb|\chapter{Code chương trình}

Nếu báo cáo không có phụ lục, sinh viên tìm và xóa dòng \verb|\chapter{Code chương trình}

Nếu báo cáo không có phụ lục, sinh viên tìm và xóa dòng \verb|\chapter{Code chương trình}

Nếu báo cáo không có phụ lục, sinh viên tìm và xóa dòng \verb|\input{text/phuluc.tex}| trong file ``thesis.tex''.
Code chương trình có thể để vào phụ lục A và các chứng minh toán học hỗ trợ có thể để vào phụ lục B.
Việc lấy code và định dạng được Latex thực hiện tự động.

Sinh viên tham khảo file ``phuluc.tex'' để biết cách đưa code vào báo cáo.
Trong lệnh \verb|\lstinputlisting|, báo cáo mẫu có đường dẫn đến file ``spectrum.m'' trong thư mục con ``Code'' nằm chung với thư mục chứa báo cáo.
Sinh viên có thể copy các file code, chẳng hạn code Matlab, vào thư mục ``Code'' và tham khảo file ``phuluc.tex'' để biết cách đưa code vào báo cáo.
Để tiện lợi hơn, sinh viên có thể chỉnh đường dẫn trong lệnh này đến thư mục chứa code của mình để Latex có thể cập nhật trực tiếp.
Để định dạng theo các ngôn ngữ lập trình khác nhau, sinh viên thay giá trị ``Maltab'' trong lệnh thành các giá trị khác phù hợp như ``C++'' hay ``Python''. 

Sau đây là ví dụ code Matlab:

\small
\lstinputlisting[language=Matlab, numbers=left, numberstyle=\small, breaklines]{code/spectrum.m}
| trong file ``thesis.tex''.
Code chương trình có thể để vào phụ lục A và các chứng minh toán học hỗ trợ có thể để vào phụ lục B.
Việc lấy code và định dạng được Latex thực hiện tự động.

Sinh viên tham khảo file ``phuluc.tex'' để biết cách đưa code vào báo cáo.
Trong lệnh \verb|\lstinputlisting|, báo cáo mẫu có đường dẫn đến file ``spectrum.m'' trong thư mục con ``Code'' nằm chung với thư mục chứa báo cáo.
Sinh viên có thể copy các file code, chẳng hạn code Matlab, vào thư mục ``Code'' và tham khảo file ``phuluc.tex'' để biết cách đưa code vào báo cáo.
Để tiện lợi hơn, sinh viên có thể chỉnh đường dẫn trong lệnh này đến thư mục chứa code của mình để Latex có thể cập nhật trực tiếp.
Để định dạng theo các ngôn ngữ lập trình khác nhau, sinh viên thay giá trị ``Maltab'' trong lệnh thành các giá trị khác phù hợp như ``C++'' hay ``Python''. 

Sau đây là ví dụ code Matlab:

\small
\lstinputlisting[language=Matlab, numbers=left, numberstyle=\small, breaklines]{code/spectrum.m}
| trong file ``thesis.tex''.
Code chương trình có thể để vào phụ lục A và các chứng minh toán học hỗ trợ có thể để vào phụ lục B.
Việc lấy code và định dạng được Latex thực hiện tự động.

Sinh viên tham khảo file ``phuluc.tex'' để biết cách đưa code vào báo cáo.
Trong lệnh \verb|\lstinputlisting|, báo cáo mẫu có đường dẫn đến file ``spectrum.m'' trong thư mục con ``Code'' nằm chung với thư mục chứa báo cáo.
Sinh viên có thể copy các file code, chẳng hạn code Matlab, vào thư mục ``Code'' và tham khảo file ``phuluc.tex'' để biết cách đưa code vào báo cáo.
Để tiện lợi hơn, sinh viên có thể chỉnh đường dẫn trong lệnh này đến thư mục chứa code của mình để Latex có thể cập nhật trực tiếp.
Để định dạng theo các ngôn ngữ lập trình khác nhau, sinh viên thay giá trị ``Maltab'' trong lệnh thành các giá trị khác phù hợp như ``C++'' hay ``Python''. 

Sau đây là ví dụ code Matlab:

\small
\lstinputlisting[language=Matlab, numbers=left, numberstyle=\small, breaklines]{code/spectrum.m}
| trong file ``thesis.tex''.
Code chương trình có thể để vào phụ lục A và các chứng minh toán học hỗ trợ có thể để vào phụ lục B.
Việc lấy code và định dạng được Latex thực hiện tự động.

Sinh viên tham khảo file ``phuluc.tex'' để biết cách đưa code vào báo cáo.
Trong lệnh \verb|\lstinputlisting|, báo cáo mẫu có đường dẫn đến file ``spectrum.m'' trong thư mục con ``Code'' nằm chung với thư mục chứa báo cáo.
Sinh viên có thể copy các file code, chẳng hạn code Matlab, vào thư mục ``Code'' và tham khảo file ``phuluc.tex'' để biết cách đưa code vào báo cáo.
Để tiện lợi hơn, sinh viên có thể chỉnh đường dẫn trong lệnh này đến thư mục chứa code của mình để Latex có thể cập nhật trực tiếp.
Để định dạng theo các ngôn ngữ lập trình khác nhau, sinh viên thay giá trị ``Maltab'' trong lệnh thành các giá trị khác phù hợp như ``C++'' hay ``Python''. 

Sau đây là ví dụ code Matlab:

\small
\lstinputlisting[language=Matlab, numbers=left, numberstyle=\small, breaklines]{code/spectrum.m}
