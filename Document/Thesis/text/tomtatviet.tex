
Trong cuộc sống hiện đại ngày nay, chất lượng cuộc sống ngày càng được nâng cao, thì vấn đề sức khỏe, đặc biệt là về các bệnh tim mạch ngày càng được quan tâm, những phương pháp mới tiến bộ hơn được sử dụng vào trong chẩn đoán và điều trị. Một phương pháp chẩn đoán hiện nay được quan tâm nghiên cứu rất nhiều đó là phương pháp chụp cắt lớp MRI. Khi chụp MRI khu vực tim sẽ được chụp nhiều lát cắt (slide), số slide này có số lượng tầm khoảng 8 - 16 nhát cắt, và để chẩn đoán thì cần phải chụp nhiều chuỗi ảnh như vậy để có thể quan sát quy trình tim đập. Để xử lý loại dữ liệu này, em sử dụng phương pháp 3D CNN kết hợp với LSTM để phân loại các chuỗi ảnh này. Đề cương này em đã cài đặt một thuật toán và train nó trên một tập dataset chuỗi ảnh MRI tim nhỏ. Thuật toán này có thể phân loại chuỗi ảnh tim là bình thường hay không với độ chính xác 98.4 \% sau 10 epochs

